\section{ระเบียบวิธี} 

\subsection{วัตถุประสงค์}  
\indent\indent
\gls{sdn} นั้นมีการใช้งานและการตั้งค่าที่หลากหลาย
ไม่ว่าจะเพื่อให้เหมาะกับ \gls{network-system} ขององค์กรหรือหน่วยงานของตนเอง ยังต้องช่วยเพิ่มประสิทธิภาพของ\gls{network-system}อีกด้วย
นอกจากนั้น \gls{sdn} ยังถูกนำมาใช้เพื่อป้องกันการโจมตีผ่าน\gls{network-system} ไม่ว่าจะเป็น Dos หรือ DDos
อย่างไรก็ดี\gls{sdn}นั้นก็คือ \gls{software} รูปแบบหนึ่ง
ดังนั้นก็จะมี\gls{software}หลายตัวที่ถูกพัฒนาขึ้นเป็น ``\gls{sdn}'' เช่น RYU OpenDaylight NOX POX

วัตถุประสงค์ของการทดลองนี้เพื่อหาประสิทธิภาพทั้งในด้านเวลาและการตรวจจับการโจมตีของ \gls{ryu} ที่ถูกรันอยู่ใน SDN Controller
พร้อมกับการใช้งาน Group table หรือ Proxy ARP ทั้งในรูปแบบ Single Controller และ Multi Controller 
\\
\subsection{การตั้งค่าการจำลอง (Simulation setup)}
\indent\indent
ให้นักศึกษา บอกถึง เครื่องมือ (Tools) ทั้งหมด ที่นักศึกษาใช้ ทดสอบระบบ เช่น ใช้  emulation Mininet จำลอง Topology แบบ Single Controller และ Multiple
Controlleres ดังภาพที่ 1.1 หรือ อะไรก็ว่าไป เป็นต้น
น่าจะกล่าวถึง Ryu Controller Famework, emulation Mininet, ภาษา Python 
\\--------------\\
\textbf{เครื่องมือและอุปกร์ที่ใช้ในการทดลองมีดังนี้} 
\begin{itemize}
    \item Oracle VM VirtualBox เวอร์ชั่น 7.0.14 ใช้สำหรับการจำลองเครื่องเซิร์ฟเวอร์
    \item ระบบปฏิบัติการ Ubuntu server 22.04.3
    \item Python 3.9.18 และ 2.7.18
    \item RYU version 4.34
    \item mininet version 2.3.1b4
    \item wireshark 3.6.2
    \item vim version 8.2.2121
\end{itemize}
\\
\begin{LARGE}
     ถึงตรงนี้ และจะเขียนคำอธิบายเพิ่มเติมสำหรับ เครื่องมือ
\end{LARGE}
\subsection{รูปแบบในการทดลอง}
\indent\indent
การทดลองมี 4 รูปแบบ ดัง (A1)-(A4) โดย (A1) และ (A3) ใช้คำสั้ง XXX ในการจำลอง Topology และ (A2) และ (A4) ใช้ภาษา Python 3.9 ในการจำลอง Topology ซึ่งทั้ง 4 รูปแบบมีการจองลองการโจมตีโดยการใช้คำสั้ง ping จาก Node-N ใดไปยัง Node-M ใด ด้วยจำนวน Package เท่าไร

\$ ryu-manager learn-sdn-with-ryu/ryu-exercises/ex7\_group\_tables.py

\$ sudo python3 (ชื่อไฟล์).py

