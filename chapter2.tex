\section{ผลลัพธ์และการอภิปราย (Results and discussion)}
\indent\indent
เหตุผลที่ต้องทำการรัน 3 รอบ ในแต่ละรูปแบบนั้น เนื่องมาจากในขั้นตอนการเตรียมเครื่องมือและซอฟต์แวร์นั้นได้มีการรันทดสอบ และพบว่าค่ามีความแกว่งในบางครั้ง โดยในการรันครั้งแรกดรอปแพ็กเกจได้ 8\% ในการรันครั้งที่ 2 อาจจะเพิ่มเป็น 30\% หรือ 35\% 
ดังนั้นจึงได้มีการออกแบบขั้นตอนการทดลองที่เหมือนกันมากที่สุดเพื่อลดปัจจัยที่อาจจะกระทบและทำให้ผลลัพธ์เปลี่ยนไปอย่างมีนัยสำคัญ นอกจากนั้นจึงได้มีการรันทดสอบ 3 รอบของแต่ละรูปแบบเพื่อลดโอกาสที่ค่าจะออกมาคลาดเคลื่อนหรือผิดไปจากที่ควรจะเป็น 
และในการรันทดสอบนั้นมีการจับแพ็กเกจด้วย wireshark ที่ s4-eth1 และ s11-eth2 ซึ่งทำให้มี 2 ค่าที่เกิดขึ้นจากทั้ง 2 ขาของการจับแพ็กเกจ ทางผู้ทดลองต้องการทำให้เป็นค่าเพียงหนึ่งค่า
จึงจะทำการรวมค่าจาก s4-eth1 และ s11-eth2 ให้เป็นค่าหนึ่งโดยการบวก และในการรันทดสอบ 3 ครั้งนั้น 
ผู้ทำการทดลองพิจารณาแล้วว่าจะเลือกค่าอัตราการดรอปแพ็กเกจที่น้อยที่สุดมาใช้ในการวิเคราะห์ ด้วยเหตุที่ว่าค่าที่มีการแกว่งนั้นมักจะเป็นค่าที่มีค่ามากหรือมากที่สุด
\\
\subsection{ผลลัพธ์ประสิทธิภาพด้านเวลา}
\indent\indent
ผลลัพธ์การรันทดสอบการโจมตีในด้านประสิทธิภาพเวลา มีผลลัพธ์เป็นดังนี้ 
Group table แบบ Single Controller (A1) มีค่า time span เท่ากับ 18.7
Group table แบบ Multi Controllers (A2) มีค่า time span เท่ากับ 21.5
Proxy Arp แบบ Single Controller (A3) มีค่า time span เท่ากับ 14.1
Proxy Arp แบบ Multi Controllers (A4) มีค่า time span เท่ากับ 22.3

\begin{figure}[h]
    \centering
    \begin{tikzpicture}
        \begin{axis}[
            title=\textbf{ประสิทธิภาพด้านเวลา},
            ylabel=time (s),
            ybar,
            ymax=30,
            ymin=0,
            nodes near coords,
            nodes near coords align={vertical},
            bar width=1em,
            symbolic x coords={A1,A2,A3,A4},
            xtick=data,
            enlarge x limits=0.5,
            ]
            \addplot[fill=blue] coordinates {
                (A1,18.7)
                (A2,21.5)
                (A3,14.1)
                (A4,22.3)
            };
           % \legend{A1,A2,A3,A4}
            % \legend{Men,Women}
        \end{axis}
    \end{tikzpicture}
    \caption{แผนภูมิแสดงผลลัพธ์ประสิทธิภาพด้านเวลา}
\end{figure}
\subsection{ผลลัพธ์ประสิทธิภาพด้านการตรวจจับการโจมตี}
\indent\indent
ผลลัพธ์การรันทดสอบการโจมตีในด้านประสิทธิภาพการตรวจจับการโจมตี มีผลลัพธ์เป็นดังนี้ 
Group table แบบ Single Controller (A1) มีค่า droped packages เท่ากับ 8.7\%
Group table แบบ Multi Controllers (A2) มีค่า droped packages เท่ากับ 20.4\%
Proxy Arp แบบ Single Controller (A3) มีค่า droped packages เท่ากับ 6.3\%
Proxy Arp แบบ Multi Controllers (A4) มีค่า droped packages เท่ากับ 19.4\%
\\
\begin{figure}[h!]
    \centering
    \begin{tikzpicture}
        \begin{axis}[
            title=\textbf{ประสิทธิภาพด้านการตรวจจับการโจมตี},
            ylabel=droped packages (\%),
            ybar,
            ymax=30,
            ymin=0,
            nodes near coords,
            nodes near coords align={vertical},
            bar width=1em,
            symbolic x coords={A1,A2,A3,A4},
            xtick=data,
            enlarge x limits=0.5,
            ]
            \addplot[fill=red] coordinates {
                (A1,8.7)
                (A2,20.4)
                (A3,6.3)
                (A4,19.4)
            };
           % \legend{A1,A2,A3,A4}
            % \legend{Men,Women}
        \end{axis}
    \end{tikzpicture}
    \caption{แผนภูมิแสดงผลลัพธ์ประสิทธิภาพด้านการตรวจจับการโจมตี}
\end{figure}
% \clearpage
\subsection{สรุปผลการทดลอง}
\indent\indent
จากผลลัพธ์ที่ได้รายงานข้างต้น จะสังเกตได้ว่าในประสิทธิภาพด้านเวลานั้นไม่ได้มีความแตกต่างกันมาก แต่ก็พอจะกล่าวได้ว่าการใช้ Proxy Arp นั้นให้ประสิทธิภาพด้านความเร็วที่มากกว่า group table รวมไปถึงการใช้ Multi Controller ก็ให้ประสิทธิภาพด้านความเร็วที่มากกว่า Single Controllers 
และหากพิจารณาระหว่าง group table และ proxy arp บน Single Controller จะเห็นได้ว่า Proxy arp ให้ประสิทธิภาพด้านความเร็วอย่างเห็นได้ชัด
แต่หากพิจารณาบน Muti Controllers ก็จะพบว่าไม่ได้มีความต่างเท่าใดนัก อาจกล่าวได้ว่าให้ประสิทธิภาพที่เท่าๆ กัน
\\\indent
ในประสิทธิภาพด้านการตรวจจับการโจมตีนั้น จะสังเกตได้ว่า ทั้ง Group table และ Proxy arp ไม่ได้มีความต่างอย่างมีนัยสำคัญ แต่ว่าสิ่งที่ต่างอย่างเห็นได้ชัดเจนคือการรันแบบ Single Controller และ Multi Controllers โดยการรันแบบ Multi Controllers จะสามารถตรวจจับการโจมตีและทำการ drop packages ได้มากกว่าการรันแบบ Single Controller 
ดังนั้นหากพิจารณาต้องการการตรวจจับและป้องกันการโจมตี การรันด้วย Multi Controllers ก็ถือเป็นตัวเลือกที่น่าสนใจ
\\
\subsection{ข้อเสนอแนะ}
\indent\indent
ในการทดลองครั้งนี้ถูกรันบนเครื่องเซิร์ฟเวอร์จำลอง และสร้าง Topology จำลองขึ้น นอกจากนั้นการออกแบบขั้นตอนการทดลองยังคงไม่รัดกุมและชัดเจนมากเท่าที่ควร 
ซึ่งอาจจะทำให้ผลลัพธ์คลาดเคลื่อนจากความเป็นจริง อาจกล่าวได้ว่าการทดลองนี้เป็นเพียงการทดลองเบื้องต้นเพียงเท่านั้น
ดังนั้นผู้อ่านจึงต้องมีความระมัดระวัง และรอบคอบในการนำไปใช้

\begin{comment}

group_single 8.7 80.4 44.2 | 44.4 | 8.7
group_single 18.7 8.4 6.8         | 18.7

group_multi 32.5 31.2 20.4 | 28.0 | 20.4
group_multi 23.5 21.8 21.5        | 21.5

arp_single 12   6.3  70.3  | 29.5 | 6.3
arp_single 19.2 14.1 7.8          | 14.1  

arp_multi 37.2 25.6 19.4   | 27.4 | 19.4
arp_multi 24.1 25.5 22.3          | 22.3

\end{comment}