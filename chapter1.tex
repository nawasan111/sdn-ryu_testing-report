\section{ระเบียบวิธี} 

\subsection{วัตถุประสงค์}  
\indent\indent
\gls{sdn} นั้นมีการใช้งานและการตั้งค่าที่หลากหลาย
ไม่ว่าจะเพื่อให้เหมาะกับ \gls{network-system} ขององค์กรหรือหน่วยงานของตน ยังต้องช่วยเพิ่มประสิทธิภาพของ\gls{network-system}อีกด้วย
นอกจากนั้น \gls{sdn} ยังถูกนำมาใช้เพื่อป้องกันการโจมตีผ่าน\gls{network-system} ไม่ว่าจะเป็น Dos หรือ DDos
อย่างไรก็ตาม\gls{sdn}นั้นก็คือ \gls{software} รูปแบบหนึ่ง
ดังนั้นก็จะมี\gls{software}หลายตัวที่ถูกพัฒนาขึ้นเป็น ``\gls{sdn}'' เช่น RYU OpenDaylight NOX POX ฯลฯ 
ตังนั้น
วัตถุประสงค์ของการทดลองนี้ก็เพื่อหาประสิทธิภาพทั้งในด้านเวลาและการตรวจจับการโจมตีของ \gls{ryu} ซึ่งเป็น\gls{sdn}ประเภทหนึ่ง ที่ถูกรันอยู่ใน SDN Controller
พร้อมกับการใช้งาน Group table หรือ Proxy ARP ทั้งในรูปแบบ Single Controller และ Multi Controller 
\\
\subsection{การตั้งค่าการจำลอง (Simulation setup)}

\textbf{เครื่องมือและอุปกร์ที่ใช้ในการทดลองมีดังนี้} 

\begin{itemize}
    \item Notebook Lenovo IdeaPad 5 14IIL05 รันด้วยระบบปฏิบัติการ Arch Linux 6.7.4-arch1-1 
    \hspace{1em}เป็นอุปกรณ์ที่ใช้สำหรับการรัน\gls{software}ทดสอบทั้งหมด
    \item GNOME Terminal Version 3.50.1 \hspace{1em}
    เป็น\gls{software}อยู่บน Notebook ที่จะใช้ในการ ssh เข้าไปใน server 
    \item Oracle VM VirtualBox เวอร์ชั่น 7.0.14\hspace{1em} 
    ใช้สำหรับการจำลองเครื่องเซิร์ฟเวอร์ขึ้นมาเป็นอีกเครื่องหนึ่งบน Notebook
     ที่กล่าวไปข้างต้น ซึ่งจะทำให้เหมือนว่ามีเครื่องเซิร์ฟเวอร์แยกเป็นอีกเครื่องหนึ่งจริงๆ
    \item ระบบปฏิบัติการ Ubuntu server 22.04.3\hspace{1em}
    คือระบบปฏิบัติการที่ถูกรันอยู่บนเครื่อง Oracle VM VirtualBox 
    \item Python version 3.9.18 และ 2.7.18 \hspace{1em}
    คือ runtime สำหรับรัน script ไฟล์ที่เขียนด้วยภาษา Python ซึ่งจะใช้ในการรัน \gls{ryu} รวมไปถึงการจำลอง Topology ของ mininet
    \item RYU version 4.34\hspace{1em}
    คือ \gls{software} ที่ได้กล่าวไปข้างต้น ซึ่งเป็น\gls{software}หลักตัวหนึ่ง ที่จะถูกใช้ในการทดลองครั้งนี้
    \item mininet version 2.3.1b4\hspace{1em}
    คือ\gls{software}ที่จะทำการจำลอง Topology ขึ้นมาภายในเครื่องจำลอง Ubuntu server ที่รันอยู่บน Oracle VM VirtualBox
    \item wireshark 3.6.2 \hspace{1em}
    คือ\gls{software}ที่จะช่วยในการดักจับข้อมูลที่วิ่งผ่าน\gls{network-system} ซึ่งจะเป็น\gls{software}ที่ใช้ในการวัดประสิทธิภาพในด้านเวลาและการป้องกันการโจมตี
    \item vim version 8.2.2121\hspace{1em}
    คือ\gls{software}ที่ใช้ในการแก้ไขไฟล์ข้อความหรือไฟล์ script ต่างๆ บนระบบปฏิบัติการ Ubuntu server
    \item Vscode 1.86.1\hspace{1em}
    เป็น\gls{software}อีกตัวหนึ่งที่ใช้ในการแก้ไขไฟล์ข้อความซึ่งมี Extensions ช่วยในการเขียน ซึ่งในการทดลองนี้จะนำมาใช้ในการแก้ไขไฟล์ script ของ Python ในกรณีที่ต้องการความแม่นยำหรือแก้ปัญหาที่ซับซ้อน
\end{itemize}

\subsection{รูปแบบในการทดลอง}
\indent\indent
ก่อนที่จะทำการทดลองนั้นต้องทำการเตรียม python scripts ที่ใช้ในการสร้าง topology จำลองบน mininet
ซึ่งต้องสร้างทั้งแบบ Single Controller และ Multi Controllers โดยสามารถใช้ mininet gui สร้างได้ ตามภาพด้านล่างนี้
หรือสามารถดาวน์โหลดไฟล์ scripts ได้ที่ http://projectcs.sci.ubu.ac.th/nawasan/sdn-topo-mininet
\\
\begin{figure}[h!]
    \centering
    \includegraphics*[width=20em]{single-edit.png}
    \caption{Topology แบบ Single Controller}
    \label{img:topo_cs}
\end{figure}

\begin{figure}[h!]
    \centering
    \includegraphics*[width=20em]{multi-edit.png}
    \caption{Topology แบบ Multi Controller}
    \label{img:topo_cm}
\end{figure}

ในการทดลองครั้งนี้มี 4 รูปแบบ ซึ่งจะกำหนดให้มีความต่างเฉพาะที่กำหนด และขั้นตอนวิธียังคงเหมือนกัน
โดยสิ่งที่ต้องเตรียมคือ Terminal ให้ทำการเปิดขึ้นมา 3 หน้าต่าง และทำการ ssh เข้าไปที่เซิร์ฟเวอร์จำลองให้เรียบร้อย 
โดยหน้าต่างที่ 1 จะใช้สำหรับการรัน \gls{ryu} ทั้งแบบ group table และ proxy arp 
หน้าต่างที่ 2 จะใช้สำหรับการรัน topology จำลอง ทั้งแบบ Single Controller และ Multi Controllers 
และให้ h2 โจมตีไปยัง h12
หน้าต่างที่ 3 จะใช้สำหรับการรัน wireshark สำหรับตรวจจับแพ็กเกจ
โดยจะมีขั้นตอนดังนี้

\begin{enumerate}
    \item หน้าต่างที่ 1 ทำการรันคำสั่ง
    \$ ryu-manager learn-sdn-with-ryu/ryu-exercises/<python script> \\โดย python script จะแทนด้วย ex7\_group\_tables.py และ ex8\_arp\_proxy.py 
    \\ซึ่งได้มาจาก https://github.com/knetsolutions/learn-sdn-with-ryu.git
    \item หน้าต่างที่ 2 ทำการรันคำสั่ง
    \$ sudo python3 <python script> \\
    โดย python script จะแทนด้วย single\_topo.py และ multi\_topo.py 
    \item หน้าต่างที่ 3 ทำการรันคำสั่ง
    \$ sudo wireshark \\
    โดยจับที่ขา s4-eth0 และ s11-eth2 ปริมาณ 1,000,000 แพ็กเกจ
    \item หน้าต่างที่ 2 ให้ h2 โจมตี h12 ด้วยคำสั่ง \$ mininet> h2 hping3 h12 -S --flood -V
    \item เมื่อทำการจับแพ็กเกจครบ 1,000,000 แพ็กเกจแล้ว ทำการเก็บข้อมูลวิเคราะห์จาก wireshark จากนั้นให้ทำการหยุดรันในทุกๆ หน้าต่าง หากจะทำการทดลองรอบถัดไปต้องรันขั้นตอนใหม่ตั้งแต่ขั้นตอนที่ 1.
\end{enumerate}

ในการทดลองครั้งนี้จะมี 4 รูปแบบ โดยทั้ง 4 รูปแบบจะทำการรัน 3 รอบ คือ 

\begin{enumerate}
    \item (A1) group table แบบ Single Controller \\
    หน้าต่างที่ 1 จะทำการรัน ryu ด้วย script ไฟล์ที่ชื่อ ex7\_group\_tables.py 
    และ หน้าต่างที่ 2 ทำการรัน single-topo.py
    \item (A2) group table แบบ Multi Controllers \\
    หน้าต่างที่ 1 จะทำการรัน ryu ด้วย script ไฟล์ที่ชื่อ ex7\_group\_tables.py
    และ หน้าต่างที่ 2 ทำการรัน multi-topo.py
    \item (A3) proxy arp แบบ Single Controller \\
    หน้าต่างที่ 1 จะทำการรัน ryu ด้วย script ไฟล์ที่ชื่อ ex8\_arp\_proxy.py
    และ หน้าต่างที่ 2 ทำการรัน single-topo.py
    \item (A4) proxy arp แบบ Multi Controllers \\
    หน้าต่างที่ 1 จะทำการรัน ryu ด้วย script ไฟล์ที่ชื่อ ex8\_arp\_proxy.py
    และ หน้าต่างที่ 2 ทำการรัน multi-topo.py
 
\end{enumerate}
