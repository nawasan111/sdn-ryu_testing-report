\section{ผลลัพธ์และการอภิปราย (Results and discussion)}
เหตุผลที่ต้องทำการรัน 3 รอบ ในแต่ละเงื่อนไข เนื่องมาจากในขั้นตอนการเตรียมเครื่องมือและซอฟต์แวร์นั้นได้มีการรันทดสอบ และพบว่าค่ามีความแกว่งในบางครั้ง โดยในการรันครั้งแรกดรอปแพ็กเกจได้ 8\% ในการรันครั้งที่ 2 อาจจะเพิ่มเป็น 30\% หรือ 35\% 
ดังนั้นจึงได้มีการออกแบบขั้นตอนการทดลองที่เหมือนกันมากที่สุดเพื่อลดปัจจัยที่อาจจะกระทบและทำให้ผลลัพธ์เปลี่ยนไปอย่างมีนัยสำคัญ นอกจากนั้นจึงได้มีการรันทดสอบ 3 รอบของแต่ละรูปแบบเพื่อลดโอกาสที่ค่าจะออกมาคลาดเคลื่อนหรือผิดไปจากที่ควรจะเป็น 
และในการรันทดสอบนั้นมีการจับแพ็กเก็ตด้วย wireshark ที่ s4-eth1 และ s11-eth2 ซึ่งทำให้มี 2 ค่าที่เกิดขึ้นจากทั้ง 2 ขาของการจับแพ็กเกจ ทางผู้ทดลองต้องการทำให้เป็นค่าเพียงหนึ่งค่า
จึงจะทำการรวมค่าจาก s4-eth1 และ s11-eth2 ให้เป็นค่าหนึ่งโดยการบวก และในการรันทดสอบ 3 ครั้งนั้น 
ผู้ทำการทดลองพิจารณาแล้วว่าจะเลือกค่าอัตราการดรอปแพ็กเกจที่เป็นค่ากลางมาใช้ในการวิเคราะห์
\\
\subsection{ผลลัพธ์การทดลอง}
\subsubsection*{ผลลัพธ์ด้านเวลา}

%%%%%
%
%% Here
%
%%%%%
ในการทดสอบทั้ง 2 รูปแบบมีการจับแพ็กเก็ตหลายระดับ จึงอาจจะทำให้แพ็กเก็จจำนวนมากกว่าจะใช้เวลามากกว่า
จึงได้ใช้สูตรคำนวนเพื่อให้สามารถเทียบวัดได้ โดยใช้สูตร
$$P=t \times \frac{m}{d}$$

โดยที่ P คือค่าคะแนน t คือเวลาที่ได้จากการวัด m คือจำนวนแพ็กเก็ตมากที่สุดที่ทำการวัด d คือจำนวนแพ็กเก็ตที่วัดในรอบนั้น

\begin{figure}[h!]
    \centering
    \begin{tikzpicture}
        \begin{axis}[
            title=\textbf{ประสิทธิภาพด้านเวลา การทดลองรูปแบบที่ 1},
             width=0.8\textwidth,
            height=0.5\textwidth,
            x tick label style={
		    /pgf/number format/1000 sep=},
            legend style={at={(1,1)},
	        anchor=north east},
            xlabel=packet,
            ylabel=P (ยิ่งน้อยยิ่งดี),
            % ybar interval=0.7,
            % ybar,
            ymax=50,
            ymin=0,
            symbolic x coords={
            100000,
            250000,
            500000,
            750000,
            1000000
            },
            xtick=data,
            enlarge x limits=0.15,
            ]
            \addplot coordinates {
                % group table single
                (100000,26.89)
                (250000,20.956)
                (500000,21.614)
                (750000,18.601)
                (1000000,20.595)
            };
            \addplot coordinates {
                % group table multiple
                (100000,19.75)
                (250000,8.724)
                (500000,14.478)
                (750000,33.437)
                (1000000,33.052)
            };
            \addplot coordinates {
                % arp proxy single
                (100000,28.49)
                (250000,20.104)
                (500000,21.776)
                (750000,20.478)
                (1000000,21.294)
            };
             \addplot coordinates {
                % arp proxy multiple
                (100000,42.3)
                (250000,35.8)
                (500000,26.944)
                (750000,29.342)
                (1000000,21.489)
            };
           \legend{
            A1,A2,A3,A4
          }
        \end{axis}
    \end{tikzpicture}
    \caption{แผนภูมิแสดงผลลัพธ์ประสิทธิภาพด้านเวลาของการทดลองรูปแบบที่ 1}
    \label{img:graph_time_1}
\end{figure}

จากรูปภาพที่ \ref{img:graph_time_1} (A1) มีประสิทธิภาพไม่ต่างกันอย่างมีนัยสำคัญเมื่อแพ็กเก็ตเพิ่มขึ้น แต่มีแนวโน้มดีขึ้นเมื่อแพ็กเก็ตเพิ่มขึ้น
(A2) มีแนวโน้มมีประสิทธิภาพน้อยลงเมื่อแพ็กเก็ตเพิ่มขึ้น แต่ไม่สม่ำเสมอ 
(A3) ประสิทธิภาพดีขึ้นเล็กน้อยอย่างไม่มีนัยสำคัญเมื่อแพ็กเก็ตเพิ่มขึ้น 
(A4) มีประสิทธิภาพดีขึ้นเมื่อแพ็กเก็ตเพิ่มขึ้น และมีแนวโน้มประสิทธิภาพจะดีขึ้นอีก เมื่อแพ็กเก็ตเพิ่มขึ้น

\begin{figure}[h!]
    \centering
    \begin{tikzpicture}
        \begin{axis}[
            title=\textbf{ประสิทธิภาพด้านเวลา การทดลองรูปแบบที่ 2},
             width=0.8\textwidth,
            height=0.5\textwidth,
            x tick label style={
		    /pgf/number format/1000 sep=},
            legend style={at={(1,1)},
	        anchor=north east},
            xlabel=packet,
            ylabel=P (ยิ่งน้อยยิ่งดี),
            % ybar interval=0.7,
            % ybar,
            ymax=50,
            ymin=0,
            symbolic x coords={
            1000000,
            2000000
            },
            xtick=data,
            enlarge x limits=0.15,
            ]
            \addplot coordinates {
                % group table single
                (1000000,27.994)
                (2000000,21.84)
            };
            \addplot coordinates {
                % group table multiple
                (1000000,29.55)
                (2000000,13.319)
            };
            \addplot coordinates {
                % arp proxy single
                (1000000,21.66)
                (2000000,23.125)
            };
             \addplot coordinates {
                % arp proxy multiple
                (1000000,18.478)
                (2000000,17.851)
            };
           \legend{
            A1,A2,A3,A4
          }
        \end{axis}
    \end{tikzpicture}
    \caption{แผนภูมิแสดงผลลัพธ์ประสิทธิภาพด้านเวลาของการทดลองรูปแบบที่ 2}
    \label{img:graph_time_2}
\end{figure}

จากรูปภาพที่ \ref{img:graph_time_2} (A1) มีประสิทธิภาพดีขึ้นเมื่อแพ็กเก็ตเพิ่มขึ้น
(A2) มีประสิทธิภาพดีขึ้นอย่างเห็นได้ชัดเมื่อแพ็กเก็ตเพิ่มขึ้น
(A3) ประสิทธิภาพเพิ่มขึ้นเล็กน้อย แต่ไม่มีนัยสำคัญ
(A4) ไม่มีการเปลี่ยนแปลงอย่างเห็นได้ชัด อาจกล่าวได้ว่าประสิทธิภาพเท่าเดิม

\subsubsection*{ผลลัพธ์ประสิทธิภาพด้านการตรวจจับการโจมตี}

% attack table 
\begin{figure}[h!]
    \centering
    \begin{tikzpicture}
        \begin{axis}[
            title=\textbf{ประสิทธิการตรวจจับการโจมตี การทดลองรูปแบบที่ 1},
             width=0.8\textwidth,
            height=0.5\textwidth,
            x tick label style={
		    /pgf/number format/1000 sep=},
            legend style={at={(1,1)},
	        anchor=north east},
            xlabel=packet,
            ylabel=droped packet (\%),
            % ybar interval=0.7,
            % ybar,
            ymax=30,
            ymin=0,
            symbolic x coords={
            100000,
            250000,
            500000,
            750000,
            1000000
            },
            xtick=data,
            enlarge x limits=0.15,
            ]
            \addplot coordinates {
                % group table single
                (100000,21.6)
                (250000,17.2)
                (500000,14.9)
                (750000,5.8)
                (1000000, 11.3)
            };
            \addplot coordinates {
                % group table multiple
                (100000,9.8)
                (250000,4.4)
                (500000,8)
                (750000,4.5)
                (1000000,4)
            };
            \addplot coordinates {
                % arp proxy single
                (100000,20.9)
                (250000,10.7)
                (500000,7.5)
                (750000,7.7)
                (1000000,8)
            };
             \addplot coordinates {
                % arp proxy multiple
                (100000,8.1)
                (250000,10.9)
                (500000,5.4)
                (750000,9.3)
                (1000000,6.1)
            };
           \legend{
            A1,A2,A3,A4
          }
        \end{axis}
    \end{tikzpicture}
    \caption{แผนภูมิแสดงผลลัพธ์ประสิทธิภาพด้านการตรวจจับการโจมตีของการทดลองรูปแบบที่ 1}
    \label{img:graph_atk_1}
\end{figure}

จากรูปที่ \ref*{img:graph_atk_1} จะเห็นว่า (A1) มีประสิทธิภาพการตรวจจับที่น้อยลงเมื่อแพ็กเก็ตเพิ่มขึ้น
(A2) มีประสิทธิภาพไม่คงที่ในแต่ละจำนวนแพ็กเก็ต และไม่มีความต่างอย่างมีนัยสำคัญ
(A3) มีประสิทธิภาพน้อยลงเมื่อแพ็กเก็ตเพิ่มขึ้น แต่เมื่อถึงจุดหนึ่งประสิทธิภาพจะไม่ลดลงอีก
(A4) มีประสิทธิภาพไม่คงที่ และไม่มีแนวโน้มจะลดลงหรือเพิ่มขึ้น

\begin{figure}[h!]
    \centering
    \begin{tikzpicture}
        \begin{axis}[
            title=\textbf{ประสิทธิการตรวจจับการโจมตี การทดลองรูปแบบที่ 2},
             width=0.8\textwidth,
            height=0.5\textwidth,
            x tick label style={
		    /pgf/number format/1000 sep=},
            legend style={at={(1,1)},
	        anchor=north east},
            xlabel=packet,
            ylabel=droped packet (\%),
            % ybar interval=0.7,
            % ybar,
            ymax=30,
            ymin=0,
            symbolic x coords={
            1000000,
            2000000
            },
            xtick=data,
            enlarge x limits=0.15,
            ]
            \addplot coordinates {
                % group table single
                (1000000, 8.1)
                (2000000, 10.5)
            };
            \addplot coordinates {
                % group table multiple
                (1000000,12.5)
                (2000000,21)
            };
            \addplot coordinates {
                % arp proxy single
                (1000000,12.4)
                (2000000,16.5)
            };
             \addplot coordinates {
                % arp proxy multiple
                (1000000,20.3)
                (2000000, 12.4)
            };
           \legend{
            A1,A2,A3,A4
          }
        \end{axis}
    \end{tikzpicture}
    \caption{แผนภูมิแสดงผลลัพธ์ประสิทธิภาพด้านการตรวจจับการโจมตีของการทดลองรูปแบบที่ 2}
    \label{img:graph_atk_2}
\end{figure}

จากรูปภาพที่ \ref{img:graph_atk_2} (A1) มีประสิทธิภาพเพิ่มขึ้นเมื่อแพ็กเก็ตเพิ่มขึ้น และมีแนวโน้มประสิทธิภาพเพิ่มขึ้น
(A2) ประสิทธิภาพเพิ่มขึ้นอย่างเห็นได้ชัด และมีแนวโน้มอย่างสูงว่าประสิทธิภาพจะเพิ่มขึ้นอีกหากแพ็กเก็ตเพิ่มขึ้น
(A3) มีประสิทธิภาพเพิ่มขึ้นเมื่อแพ็กเก็ตเพิ่มขึ้น มีแนวโน้มเล็กน้อยว่าประสิทธิภาพจะเพิ่มขึ้นอีก
(A4) มีประสิทธิภาพลดลงเมื่อแพ็กเก็ตเพิ่มขึ้น ซึ่งสวนทางกับเงื่อนไขอื่นอย่างเห็นได้ชัด
\\
% \clearpage
\subsection{สรุปผลการทดลอง}

จากที่ได้กล่าวไปว่า การที่ต้องทดลอง 2 รูปแบบ ก็เพื่อที่จะเปรียบเทียบและเพิ่มความถูกต้อง
เป็นต้นว่า หากทั้งสองรูปแบบสอดคล้องกัน ก็เป็นที่น่าเชื่อว่าผลการทดลองนั้นออกมาถูกต้อง
แต่หากทั้งสองรูปแบบออกมาขัดแย้งกัน ก็เป็นไปได้ว่าการทดลองนี้ไม่ถูกต้อง หรือมีข้อผิดพลาด
ดังนั้น ในการที่จะสามารถบอกได้ชัดเจนเกี่ยวกับผลทดลองนั้น จำเป็นต้องดูทั้ง 2 รูปแบบประกอบกัน

โดยในประสิทธิภาพด้านเวลานั้น จะเห็นได้อย่างชัดเจนว่า (A2) นั้นผลออกมาขัดแย้งกันอย่างชัดเจน จึงกล่าวได้ว่า ผลลัพธ์การทดสอบของ (A2) นั้นมีความน่าเชื่อถือต่ำ
หรืออาจจะมีข้อผิดพลาดในการทดลอง จึงไม่อาจจะนำ (A2) ไปใช้ในการวิเคราะห์อื่นๆ ได้
ส่วนในผลลัพธ์อื่นๆ นั้น จะเห็นว่าทั้ง 2 รูปแบบไม่ได้ขัดแย้งกันอย่างชัดเจน ผลลัพธ์จึงมีความน่าเชื่อถือ
โดย (A4) นั้นมีประสิทธิภาพที่ดีขึ้นเมื่อแพ็กเก็ตมีจำนวนเพิ่มขึ้น ส่วนที่เหลือนั้นไม่ได้มีความต่างอย่างชัดเจน
สามารถกล่าวได้ว่า (A4) ให้ประสิทธิภาพด้านเวลาดีที่สุด

ในประสิทธิภาพด้านการตรวจจับการทดลองนั้น จะเห็นได้อย่างชัดเจนเช่นเดียวกับประสิทธิภาพด้านเวลา ว่า (A2) นั้นมีผลลัพธ์ที่ขัดแย้งกันจากทัง 2 รูปแบบ ดังนั้น (A2) จึงมีความน่าเชื่อถือต่ำ
ในส่วน (A1) นั้น แม้ดูมีแนวโน้มว่าจะขัดแย้งกัน แต่เมื่อพิจารณาผลลัพธ์ในรูปแบบที่ 2 แล้ว ค่ามีการเพิ่มขึ้นเล็กน้อยเพียงเท่านั้น ทำให้ยังไม่ได้มีความขัดแย้งอย่างเห็นได้ชัด
ส่วนอื่นๆ นั้น ถือว่าค่อนข้างมีความสอดคล้องกัน โดยที่ไม่ได้มีข้อใดมีประสิทธิภาพที่ดีกว่าอย่างเห็นได้ชัด หากดูตามกราฟแล้ว พอจะพูดได้ว่า (A3) ให้ประสิทธิภาพโดยรวมที่ดีที่สุด

ประสิทธิภาพด้านการตรวจจับการโจมตีนั้น แม้จะไม่ได้มีความขัดแย้งกันอย่างชัดเจน แต่ผลลัพธ์ก็ไม่ได้เป็นไปในทิศทางเดียวกันมากนัก ดังนั้นจึงต้องมีความระมัดระวังมากเป็นพิเศษในการนำผลลัพธ์ด้านการตรวจจับแพ็กเก็ตไปวิเคราะห์ โดยหากต้องการนำข้อมูลใดๆ ไปวิเคราะห์หรือประกอบการตัดสินใจ แนะนำให้พิจารณาเลือกรูปแบบที่ 2 เป็นอันดับแรก
เพราะมีช่วงจำนวนการจับแพ็กเก็ตมากกว่า
\\
\subsection{ข้อเสนอแนะ}

ในการทดลองนี้ แม้ผู้ทำการทดลองจะออกแบบการทดลองให้มีความรัดกุม และเข้มงวดมากที่สุด
แต่ก็หลีกเลี่ยงไม่ได้ที่จะมีข้อผิดพลาด หรือผลลัพธ์ที่คลาดเคลื่อน แม้แต่ผลลัพธ์ที่ไม่สอดคล้องเองก็ตาม
อย่างไรก็ดี หากต้องการนำผลการทดลองไปใช้ ควรทำการทดสอบและวัดผลด้วยตัวเอง ตามวิธีที่เขียนไว้ หรืออาจจะวิธีการอื่นที่ดีกว่า ซึ่งอาจจะเพิ่มความรัดกุมขึ้นอีก
หรือเพิ่มความแม่นยำของผลลัพธ์ โดยอิงตามทฤษฎีมากขึ้น และควบคุมตัวแปรได้อย่างเรียบร้อย