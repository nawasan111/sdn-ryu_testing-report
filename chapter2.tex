\section{ผลลัพธ์และการอภิปราย (Results and discussion)}
เหตุผลที่ต้องทำการรัน 3 รอบ ในแต่ละเงื่อนไข เนื่องมาจากในขั้นตอนการเตรียมเครื่องมือและซอฟต์แวร์นั้นได้มีการรันทดสอบ และพบว่าค่ามีความแกว่งในบางครั้ง โดยในการรันครั้งแรกดรอปแพ็กเกจได้ 8\% ในการรันครั้งที่ 2 อาจจะเพิ่มเป็น 30\% หรือ 35\% 
ดังนั้นจึงได้มีการออกแบบขั้นตอนการทดลองที่เหมือนกันมากที่สุดเพื่อลดปัจจัยที่อาจจะกระทบและทำให้ผลลัพธ์เปลี่ยนไปอย่างมีนัยสำคัญ นอกจากนั้นจึงได้มีการรันทดสอบ 3 รอบของแต่ละรูปแบบเพื่อลดโอกาสที่ค่าจะออกมาคลาดเคลื่อนหรือผิดไปจากที่ควรจะเป็น 
และในการรันทดสอบนั้นมีการจับแพ็กเก็ตด้วย wireshark ที่ s4-eth1 และ s11-eth2 ซึ่งทำให้มี 2 ค่าที่เกิดขึ้นจากทั้ง 2 ขาของการจับแพ็กเกจ ทางผู้ทดลองต้องการทำให้เป็นค่าเพียงหนึ่งค่า
จึงจะทำการรวมค่าจาก s4-eth1 และ s11-eth2 ให้เป็นค่าหนึ่งโดยการบวก และในการรันทดสอบ 3 ครั้งนั้น 
ผู้ทำการทดลองพิจารณาแล้วว่าจะเลือกค่าอัตราการดรอปแพ็กเกจที่เป็นค่ากลางมาใช้ในการวิเคราะห์
\\
\subsection{ผลลัพธ์การทดลอง}
\subsubsection*{ผลลัพธ์ด้านเวลา}

%%%%%
%
%% Here
%
%%%%%
ในการทดสอบทั้ง 2 รูปแบบมีการจับแพ็กเก็ตหลายระดับ จึงอาจจะทำให้แพ็กเก็จจำนวนมากกว่าจะใช้เวลามากกว่า
จึงได้ใช้สูตรคำนวนเพื่อให้สามารถเทียบวัดได้ โดยใช้สูตร
$$P=t \times \frac{m}{d}$$

โดยที่ P คือค่าคะแนน t คือเวลาที่ได้จากการวัด m คือจำนวนแพ็กเก็ตมากที่สุดที่ทำการวัด d คือจำนวนแพ็กเก็ตที่วัดในรอบนั้น

\begin{figure}[h!]
    \centering
    \begin{tikzpicture}
        \begin{axis}[
            title=\textbf{ประสิทธิภาพด้านเวลา การทดลองรูปแบบที่ 1},
             width=0.8\textwidth,
            height=0.5\textwidth,
            x tick label style={
		    /pgf/number format/1000 sep=},
            legend style={at={(1,1)},
	        anchor=north east},
            xlabel=packet,
            ylabel=P (ยิ่งน้อยยิ่งดี),
            % ybar interval=0.7,
            % ybar,
            ymax=50,
            ymin=0,
            symbolic x coords={
            100000,
            250000,
            500000,
            750000,
            1000000
            },
            xtick=data,
            enlarge x limits=0.15,
            ]
            \addplot coordinates {
                % group table single
                (100000,26.89)
                (250000,20.956)
                (500000,21.614)
                (750000,18.601)
                (1000000,20.595)
            };
            \addplot coordinates {
                % group table multiple
                (100000,19.75)
                (250000,8.724)
                (500000,14.478)
                (750000,33.437)
                (1000000,33.052)
            };
            \addplot coordinates {
                % arp proxy single
                (100000,28.49)
                (250000,20.104)
                (500000,21.776)
                (750000,20.478)
                (1000000,21.294)
            };
             \addplot coordinates {
                % arp proxy multiple
                (100000,42.3)
                (250000,35.8)
                (500000,26.944)
                (750000,29.342)
                (1000000,21.489)
            };
           \legend{
            A1,A2,A3,A4
          }
        \end{axis}
    \end{tikzpicture}
    \caption{แผนภูมิแสดงผลลัพธ์ประสิทธิภาพด้านเวลาของการทดลองรูปแบบที่ 1}
    \label{img:graph_time_1}
\end{figure}

จากรูปภาพที่ \ref{img:graph_time_1} (A1) มีประสิทธิภาพไม่ต่างกันอย่างมีนัยสำคัญเมื่อแพ็กเก็ตเพิ่มขึ้น แต่มีแนวโน้มดีขึ้นเมื่อแพ็กเก็ตเพิ่มขึ้น
(A2) มีแนวโน้มมีประสิทธิภาพน้อยลงเมื่อแพ็กเก็ตเพิ่มขึ้น แต่ไม่สม่ำเสมอ 
(A3) ประสิทธิภาพดีขึ้นเล็กน้อยอย่างไม่มีนัยสำคัญเมื่อแพ็กเก็ตเพิ่มขึ้น 
(A4) มีประสิทธิภาพดีขึ้นเมื่อแพ็กเก็ตเพิ่มขึ้น และมีแนวโน้มประสิทธิภาพจะดีขึ้นอีก เมื่อแพ็กเก็ตเพิ่มขึ้น

\begin{figure}[h!]
    \centering
    \begin{tikzpicture}
        \begin{axis}[
            title=\textbf{ประสิทธิภาพด้านเวลา การทดลองรูปแบบที่ 2},
             width=0.8\textwidth,
            height=0.5\textwidth,
            x tick label style={
		    /pgf/number format/1000 sep=},
            legend style={at={(1,1)},
	        anchor=north east},
            xlabel=packet,
            ylabel=P (ยิ่งน้อยยิ่งดี),
            % ybar interval=0.7,
            % ybar,
            ymax=50,
            ymin=0,
            symbolic x coords={
            1000000,
            2000000
            },
            xtick=data,
            enlarge x limits=0.15,
            ]
            \addplot coordinates {
                % group table single
                (1000000,27.994)
                (2000000,21.84)
            };
            \addplot coordinates {
                % group table multiple
                (1000000,29.55)
                (2000000,13.319)
            };
            \addplot coordinates {
                % arp proxy single
                (1000000,21.66)
                (2000000,23.125)
            };
             \addplot coordinates {
                % arp proxy multiple
                (1000000,18.478)
                (2000000,17.851)
            };
           \legend{
            A1,A2,A3,A4
          }
        \end{axis}
    \end{tikzpicture}
    \caption{แผนภูมิแสดงผลลัพธ์ประสิทธิภาพด้านเวลาของการทดลองรูปแบบที่ 2}
    \label{img:graph_time_2}
\end{figure}

จากรูปภาพที่ \ref{img:graph_time_2} (A1) มีประสิทธิภาพดีขึ้นเมื่อแพ็กเก็ตเพิ่มขึ้น
(A2) มีประสิทธิภาพดีขึ้นอย่างเห็นได้ชัดเมื่อแพ็กเก็ตเพิ่มขึ้น
(A3) ประสิทธิภาพเพิ่มขึ้นเล็กน้อย แต่ไม่มีนัยสำคัญ
(A4) ไม่มีการเปลี่ยนแปลงอย่างเห็นได้ชัด อาจกล่าวได้ว่าประสิทธิภาพเท่าเดิม

\subsubsection*{ผลลัพธ์ประสิทธิภาพด้านการตรวจจับการโจมตี}

% attack table 
\begin{figure}[h!]
    \centering
    \begin{tikzpicture}
        \begin{axis}[
            title=\textbf{ประสิทธิการตรวจจับการโจมตี การทดลองรูปแบบที่ 1},
             width=0.8\textwidth,
            height=0.5\textwidth,
            x tick label style={
		    /pgf/number format/1000 sep=},
            legend style={at={(1,1)},
	        anchor=north east},
            xlabel=packet,
            ylabel=droped packet (\%),
            % ybar interval=0.7,
            % ybar,
            ymax=30,
            ymin=0,
            symbolic x coords={
            100000,
            250000,
            500000,
            750000,
            1000000
            },
            xtick=data,
            enlarge x limits=0.15,
            ]
            \addplot coordinates {
                % group table single
                (100000,21.6)
                (250000,17.2)
                (500000,14.9)
                (750000,5.8)
                (1000000, 11.3)
            };
            \addplot coordinates {
                % group table multiple
                (100000,9.8)
                (250000,4.4)
                (500000,8)
                (750000,4.5)
                (1000000,4)
            };
            \addplot coordinates {
                % arp proxy single
                (100000,20.9)
                (250000,10.7)
                (500000,7.5)
                (750000,7.7)
                (1000000,8)
            };
             \addplot coordinates {
                % arp proxy multiple
                (100000,8.1)
                (250000,10.9)
                (500000,5.4)
                (750000,9.3)
                (1000000,6.1)
            };
           \legend{
            A1,A2,A3,A4
          }
        \end{axis}
    \end{tikzpicture}
    \caption{แผนภูมิแสดงผลลัพธ์ประสิทธิภาพด้านการตรวจจับการโจมตีของการทดลองรูปแบบที่ 1}
    \label{img:graph_atk_1}
\end{figure}

จากรูปที่ \ref*{img:graph_atk_1} จะเห็นว่า (A1) มีประสิทธิภาพการตรวจจับที่น้อยลงเมื่อแพ็กเก็ตเพิ่มขึ้น
(A2) มีประสิทธิภาพไม่คงที่ในแต่ละจำนวนแพ็กเก็ต และไม่มีความต่างอย่างมีนัยสำคัญ
(A3) มีประสิทธิภาพน้อยลงเมื่อแพ็กเก็ตเพิ่มขึ้น แต่เมื่อถึงจุดหนึ่งประสิทธิภาพจะไม่ลดลงอีก
(A4) มีประสิทธิภาพไม่คงที่ และไม่มีแนวโน้มจะลดลงหรือเพิ่มขึ้น

\begin{figure}[h!]
    \centering
    \begin{tikzpicture}
        \begin{axis}[
            title=\textbf{ประสิทธิการตรวจจับการโจมตี การทดลองรูปแบบที่ 2},
             width=0.8\textwidth,
            height=0.5\textwidth,
            x tick label style={
		    /pgf/number format/1000 sep=},
            legend style={at={(1,1)},
	        anchor=north east},
            xlabel=packet,
            ylabel=droped packet (\%),
            % ybar interval=0.7,
            % ybar,
            ymax=30,
            ymin=0,
            symbolic x coords={
            1000000,
            2000000
            },
            xtick=data,
            enlarge x limits=0.15,
            ]
            \addplot coordinates {
                % group table single
                (1000000, 8.1)
                (2000000, 10.5)
            };
            \addplot coordinates {
                % group table multiple
                (1000000,12.5)
                (2000000,21)
            };
            \addplot coordinates {
                % arp proxy single
                (1000000,12.4)
                (2000000,16.5)
            };
             \addplot coordinates {
                % arp proxy multiple
                (1000000,20.3)
                (2000000, 12.4)
            };
           \legend{
            A1,A2,A3,A4
          }
        \end{axis}
    \end{tikzpicture}
    \caption{แผนภูมิแสดงผลลัพธ์ประสิทธิภาพด้านการตรวจจับการโจมตีของการทดลองรูปแบบที่ 2}
    \label{img:graph_atk_2}
\end{figure}

จากรูปภาพที่ \ref{img:graph_atk_2} (A1) มีประสิทธิภาพเพิ่มขึ้นเมื่อแพ็กเก็ตเพิ่มขึ้น และมีแนวโน้มประสิทธิภาพเพิ่มขึ้น
(A2) ประสิทธิภาพเพิ่มขึ้นอย่างเห็นได้ชัด และมีแนวโน้มอย่างสูงว่าประสิทธิภาพจะเพิ่มขึ้นอีกหากแพ็กเก็ตเพิ่มขึ้น
(A3) มีประสิทธิภาพเพิ่มขึ้นเมื่อแพ็กเก็ตเพิ่มขึ้น มีแนวโน้มเล็กน้อยว่าประสิทธิภาพจะเพิ่มขึ้นอีก
(A4) มีประสิทธิภาพลดลงเมื่อแพ็กเก็ตเพิ่มขึ้น ซึ่งสวนทางกับเงื่อนไขอื่นอย่างเห็นได้ชัด
\\
\subsection{ผลลัพธ์การทดลองด้านสถิติ}

\begin{figure}[h!]
    \centering
    \begin{tabular}{ |l|c|c|c|c|c|c|c|c|c| }
        \multicolumn{10}{c}{\textbf{สถิติผลการทดลองรูปแบบที่ 1}} \\
        \hline
        ประเภท & 100\,000 & 250\,000 & 500\,000 & 750\,000 & 1\,000\,000 & Mean & Median &  S.D. & C.V. \\
        \hline
        \hline  
        (A1) ด้านตรวจจับ & 21.6 & 17.2 & 14.9 & 5.8 & 11.3 & 14.16 & 14.9 & 5.983 & 0.401 \\ 
        (A1) ด้านเวลา & 26.89 & 20.956 & 21.614 & 18.6 & 20.5 & 21.7 & 20.9 & 3.094 & 0.147\\ 
        \hline
        (A2) ด้านตรวจจับ & 9.8 & 4.4 & 8 & 4.5 & 4 & 6.14 & 4.5 & 2.605 & 0.578 \\ 
        (A2) ด้านเวลา & 19.7 & 8.7 & 14.4 & 33.4 & 33 & 21.8 & 19.75 & 11.076 & 0.560  \\ 
        \hline
        (A3) ด้านตรวจจับ & 20.9 & 10.7 & 7.5 & 7.7 & 8 & 10.96 & 8 & 5.705 & 0.713 \\ 
        (A3) ด้านเวลา & 28.4 & 20.1 & 21.7 & 20.4 & 21.2 & 22.4 & 21.2 & 3.451 & 0.162 \\ 
        \hline
        (A4) ด้านตรวจจับ & 8.1 & 10.9 & 5.4 & 9.3 & 6.1 & 7.96 & 8.1 & 2.262 & 0.279 \\ 
        (A4) ด้านเวลา & 42.3 & 35.8 & 26.9 & 29.3 & 21.4 & 31.1 & 29.3 & 8.065 & 0.274 \\ 
        \hline
    \end{tabular}
    \caption{ตารางแสดงสถิติผลการทดลองรูปแบบที่ 1}
\end{figure}


\begin{figure}[h!]
    \centering
    \begin{tabular}{|l|c|c|c|c|c|c|}
        \multicolumn{7}{c}{\textbf{สถิติผลการทดลองรูปแบบที่ 2}} \\
        \hline
        ประเภท & 1\,000\,000 & 2\,000\,000 & Mean & Median &  S.D. & C.V. \\
        \hline
        \hline  
        (A1) ด้านตรวจจับ & 8.1 & 10.5 & 9.3 & 9.3 & 1.697 & 0.182 \\
        (A1) ด้านเวลา & 27.9 & 21.8 & 24.9 & 24.9 & 4.351 & 0.174 \\
        \hline
        (A2) ด้านตรวจจับ & 12.5 & 21 & 16.75 & 16.75 & 6.010 & 0.358 \\
        (A2) ด้านเวลา & 29.5 & 13.3 & 21.4 & 21.4 & 11.477 & 0.535 \\
        \hline
        (A3) ด้านตรวจจับ & 12.4 & 16.5 & 14.4 & 14.4 & 2.899 & 0.200 \\
        (A3) ด้านเวลา & 21.6 & 23.1 & 22.3 & 22.3 & 1.035 & 0.046 \\
        \hline
        (A4) ด้านตรวจจับ & 20.3 & 12.4 & 16.35 & 16.35 & 5.586 & 0.341 \\
        (A4) ด้านเวลา & 18.4 & 17.8 & 18.1 & 18.1645 & 0.443 & 0.0244 \\
        \hline
    \end{tabular}
    \caption{ตารางแสดงสถิติผลการทดลองรูปแบบที่ 2}
\end{figure}

\subsection{สรุปผลการทดลอง}

จากที่ได้กล่าวไปว่า ผู้ทดลองทำการออกแบบการทดลองให้มี 2 รูปแบบ เพื่อที่ว่าจะได้นำมาเปรียบเทียบ
ทั้งแนวโน้มความสัมพันธ์ ความขัดแย้ง หรือความต่อเนื่องเองก็ตาม เพื่อให้มั่นใจได้ว่าค่าที่ออกมาจะมีความถูกต้องมากที่สุด
นอกจากนั้นยังเป็นการสามารถดูประสิทธิภาพในช่วงละเอียดและช่วงหยาบได้ไหนจำนวนแพ็กเก็ตต่างๆ ตั่งแต่ 100,000 แพ็กเก็ต ไปจนถึง 2,000,000 แพ็กเก็ต

โดยในประสิทธิภาพด้านเวลานั้น จะเห็นได้อย่างชัดเจนว่า (A2) นั้นผลออกมาขัดแย้งกันอย่างชัดเจน แต่หากพิจารณาความต่อเนื่องแล้วก็สามารถบอกได้ว่า
ประสิทธิภาพของ (A2) นั้นลดลงในช่วงหนึ่งแสนไปจนถึงหนึ่งล้าน
จากนั้นประสิทธิภาพก็จะดีขึ้น ในส่วน (A4) นั้นมีประสิทธิภาพที่ดีขึ้นเรื่อยๆ ในช่วงระหว่างหนึ่งแสนถึงหนึ่งล้าน แต่ประสิทธิภาพไม่เปลี่ยนแปลงในระหว่างหนึ่งล้านถึงสองล้าน
ในส่วน (A1) และ (A4) นั้น ไม่ได้มีความเปลี่ยนแปลงประสิทธิภาพอย่างชัดเจน ทั้งในการทดลองรูปแบบที่หนึ่งและสอง
เมื่อพิจารณาได้ดังนั้น ก็สามารถสรุปได้ว่า การรันแบบหลาย\gls{controller} (A2, A4) ให้ประสิทธิภาพที่ดีกว่าการรันแบบหนึ่ง\gls{controller}
และเมื่อพิจารณาระหว่าง group table และ proxy arp บนหลาย\gls{controller} จะพบว่า group table ให้ประสิทธิภาพด้านเวลาที่ดีกว่า proxy arp

ในด้านประสิทธิภาพการตรวจจับการโจมตีนั้น จะเห็นได้ว่า (A2) ผลจากทั้งสองรูปแบบออกมาขัดแย้งกัน แต่หากพิจารณาถึงความต่อเนื่องแล้วก็จะเห็นได้ว่า
ค่ายังมีความน่าเชื่อถืออยู่ โดยประสิทธิภาพจะเริ่มเปลี่ยนแปลงอย่างมีนัยสำคัญเมื่อจำนวนแพ็กเก็ตถึงหนึ่งล้านไปจนถึงสองล้าน
และโดยส่วนใหญ่นั้น ประสิทธิภาพก็จะเพิ่มขึ้นในช่วงหนึ่งล้านถึงสองล้าน มีเพียง (A4) เท่านั้นที่ประสิทธิภาพลดลง นอกจากนั้นผลลัพธ์ (A4) จากทั้งสองรูปแบบยังไม่มีความต่อเนื่องกัน
เมื่อพิจารณาตามผลลัพธ์ จะเห็นได้ว่าค่าจากทั้งสองรูปแบบไม่ได้ชี้อย่างชัดเจนว่ารูปแบบไหนดีที่สุด ดังนั้นจึงจำเป็นต้องดูค่าสถิติประกอบ
โดยจะพบว่าผลการทดลองรูปแบบที่สองมีความน่าเชื่อถือมากกว่ารูปแบบที่หนึ่ง เมื่ออิงตามผลการทดลองรูปแบบที่สองจะพบว่า
การรันแบบหลาย\gls{controller}ให้ประสิทธิภาพโดยรวมที่ดีกว่า นอกจากนั้น group table ยังให้ประสิทธิภาพที่ดีกว่า proxy arp 
\\
\subsection{ข้อเสนอแนะ}

ในการทดลองนี้ ผู้อ่านจะนำไปใช้ได้ดีเมื่อพิจารณาร่วมกับค่าสถิติ นอกจากนั้นจะเป็นการดีกว่าหากผู้อ่านได้ทำการทดลองซ้ำเพื่อตรวจสอบความถูกต้องแม่นยำ
นอกจากนี้ผู้อ่านต้องมีความระมัดระวังในการนำไปใช้ เนื่องจากว่าการทดลองนี้ถูกทำอยู่บนเครื่องจำลอง ทั้งที่เป็น\gls{server} \gls{controller} หรืออื่นใดก็ตาม
ดังนั้นผลลัพธ์อาจจะคลาดเคลื่อนจากการนำไปใช้บนอุปกรณ์จริงได้