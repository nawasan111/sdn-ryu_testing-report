\section{ระเบียบวิธี} 

\subsection{วัตถุประสงค์}  
\indent\indent
\gls{sdn} นั้นมีการใช้งานและการตั้งค่าที่หลากหลาย
ไม่ว่าจะเพื่อให้เหมาะกับ \gls{network-system} ขององค์กรหรือหน่วยงานของตน ยังต้องช่วยเพิ่มประสิทธิภาพของ\gls{network-system}อีกด้วย
นอกจากนั้น \gls{sdn} ยังถูกนำมาใช้เพื่อป้องกันการโจมตีผ่าน\gls{network-system} ไม่ว่าจะเป็น Dos หรือ DDos
อย่างไรก็ตาม\gls{sdn}นั้นก็คือ \gls{software} รูปแบบหนึ่ง
ดังนั้นก็จะมี\gls{software}หลายตัวที่ถูกพัฒนาขึ้นเป็น ``\gls{sdn}'' เช่น RYU OpenDaylight NOX POX ฯลฯ 
ดังนั้น
วัตถุประสงค์ของการทดลองครั้งนี้ก็เพื่อหาประสิทธิภาพทั้งในด้านเวลาและการตรวจจับการโจมตีของ \gls{ryu} ซึ่งเป็น\gls{sdn}ประเภทหนึ่ง ที่ถูกรันอยู่บน SDN Controller
พร้อมกับการใช้งาน Group table หรือ Proxy ARP ทั้งในรูปแบบ Single Controller และ Multi Controllers
\\
\subsection{การตั้งค่าการจำลอง (Simulation setup)}

\textbf{เครื่องมือและอุปกร์ที่ใช้ในการทดลองมีดังนี้} 

\notebook รุ่น Lenovo IdeaPad 5 14IIL05 รันด้วยระบบปฏิบัติการ Arch Linux 6.7.4-arch1-1 
เป็นอุปกรณ์ที่ใช้สำหรับการรัน\gls{software}ทดสอบทั้งหมด

GNOME Terminal Version 3.50.1 
เป็น\gls{software}อยู่บน\notebook ที่จะใช้ในการ ssh เข้าไปใน server เพื่อทำการสั่งการทำงานต่างๆ

Oracle VM VirtualBox เวอร์ชั่น 7.0.14 
ใช้สำหรับการจำลอง \gls{server} ขึ้นมาเป็นอีกเครื่องหนึ่งบน\notebook
ที่กล่าวไปข้างต้น ซึ่งจะทำให้เสมือนว่ามี\gls{server}เพิ่มขึ้นมาอีกเครื่องหนึ่ง

ระบบปฏิบัติการ Ubuntu server 22.04.3
คือระบบปฏิบัติการที่ถูกรันอยู่บน\gls{server}ที่จำลองอยู่บน Oracle VM VirtualBox 

Python version 3.9.18 และ 2.7.18 
คือรันไทม์สำหรับรันสคริปต์ไฟล์ที่เขียนด้วยภาษา Python ซึ่งจะใช้ในการรัน \gls{ryu} รวมไปถึงการจำลอง Topology ของ mininet

RYU version 4.34
คือ \gls{sdn} ที่ได้กล่าวไปข้างต้น ซึ่งเป็น\gls{software}หลักตัวหนึ่ง ที่จะถูกใช้ในการรันเชื่อมต่อกับ \gls{controller} บน mininet

mininet version 2.3.1b4
คือ\gls{software}ที่จะทำการจำลอง Topology ที่ใช้ในการทดสอบขึ้นมาภายในเครื่องจำลอง Ubuntu server ที่รันอยู่บน Oracle VM VirtualBox

wireshark 3.6.2
คือ\gls{software}ที่จะช่วยในการดักจับข้อมูลที่วิ่งผ่าน\gls{network-system} ซึ่งจะเป็น\gls{software}ที่ใช้ในการวัดประสิทธิภาพในด้านเวลาและการป้องกันการโจมตี

vim version 8.2.2121
คือ\gls{software}ที่ใช้ในการแก้ไขไฟล์ข้อความหรือไฟล์ข้อความต่างๆ บนระบบปฏิบัติการ Ubuntu server

Vscode 1.86.1
เป็น\gls{software}อีกตัวหนึ่งที่ใช้ในการแก้ไขไฟล์ข้อความซึ่งมีส่วนเสริมที่ช่วยในการเขียน ซึ่งในการทดลองนี้จะนำมาใช้ในการแก้ไขไฟล์สคริปต์ของ Python ในกรณีที่ต้องการความแม่นยำหรือแก้ปัญหาที่ซับซ้อน
\newline
\subsection{รูปแบบในการทดลอง}
\indent\indent
ในการจะทำการทดลองนั้นจำเป็นต้องมี python script ที่ใช้ในการสร้าง topology จำลองบน mininet
ซึ่งต้องสร้างทั้งแบบ Single Controller และ Multi Controllers โดยสามารถใช้ mininet gui สร้างได้ ตามภาพด้านล่างนี้
หรือสามารถดาวน์โหลดไฟล์ scripts ได้ที่ http://projectcs.sci.ubu.ac.th/nawasan/sdn-topo-mininet
\\

\begin{figure}[h!]
    \centering
    \includegraphics*[width=20em]{single-edit.png}
    \caption{Topology แบบ Single Controller}
    \label{img:topo_cs}
\end{figure}

\begin{figure}[h!]
    \centering
    \includegraphics*[width=20em]{multi-edit.png}
    \caption{Topology แบบ Multi Controllers}
    \label{img:topo_cm}
\end{figure}

\pagebreak

โดยรูปภาพที่ \ref{img:topo_cs} คือ Topology แบบมี 1 \gls{controller} โดยจะทำการกำหนดให้\gls{controller}เป็นแบบ remote 
และรูปภาพที่ \ref{img:topo_cm} คือ Topology แบบมีหลาย\gls{controller} โดยจะทำการกำหนดให้\gls{controller}เป็นแบบ remote ทั้งหมด และกำหนด port เป็น 6633 6634 และ 6635
นอกจากนั้น เมื่อสร้าง script python ขึ้นมาแล้ว หรือดาวน์โหลดไฟล์จากลิงก์ที่ให้ไว้ จะทำการตั้งชื่อไฟล์เป็นดังนี้ Topology ที่มี 1 \gls{controller}
ดังรูปภาพที่ \ref{img:topo_cs} จะทำการตั้งชื่อเป็น single-topo.py และ Topology ที่มีหลาย\gls{controller} ดังรูปภาพที่ \ref{img:topo_cm} จะทำการตั้งชื่อเป็น multi-topo.py

ในการทดลองครั้งนี้ผู้ทำการทดลองได้ออกแบบการทดลองให้มีความเข้มงวดและรัดกุมมากที่สุด เพื่อให้ผลลัพธ์ที่ได้มีความคลาดเคลื่อนน้อยที่สุด
โดยผู้ทำการทดลองจะออกแบบการทดลองโดยให้มี 2 รูปแบบ 4 เงื่อนไข ดังนี้


เงื่อนไขที่ 1. (A1) group table แบบ 1 \gls{controller}\\
โดยจะทำการรัน ryu-manager ด้วย script ไฟล์ที่ชื่อ ex7\_group\_tables.py 
และทำการรันไฟล์ชื่อ single-topo.py ด้วย python เพื่อจำลอง Topology แบบ 1 \gls{controller}


เงื่อนไขที่ 2. (A2) group table แบบหลาย\gls{controller} \\
โดยจะทำการรัน ryu-manager ด้วย script ไฟล์ที่ชื่อ ex7\_group\_tables.py 
และทำการรันไฟล์ชื่อ multi-topo.py ด้วย python เพื่อจำลอง Topology แบบหลาย\gls{controller}


เงื่อนไขที่ 3. (A3) proxy arp แบบ 1 \gls{controller}\\
โดยจะทำการรัน ryu-manager ด้วย script ไฟล์ที่ชื่อ ex8\_arp\_proxy.py
และทำการรันไฟล์ชื่อ single-topo.py ด้วย python เพื่อจำลอง Topology แบบ 1 \gls{controller}

เงื่อนไขที่ 4. (A4) proxy arp แบบหลาย\gls{controller} \\
โดยจะทำการรัน ryu-manager ด้วย script ไฟล์ที่ชื่อ ex8\_arp\_proxy.py 
และทำการรันไฟล์ชื่อ multi-topo.py ด้วย python เพื่อจำลอง Topology แบบหลาย\gls{controller}

รูปแบบที่ 1. คือ ทำการทดสอบโดยจับแพ็กเก็ตทีละจำนวน 100,000 250,000 500,000 750,000 และ 1,000,000 แพ็กเก็ต
จากนั้นนำมาวิเคราะห์เปรียบเทียบหาประสิทธิภาพด้านเวลาและการตรวจจับการโจมตี

รูปแบบที่ 2. คือ ทำการทดสอบโดยจำแพ็กเก็ตจำนวน 1,000,000 และ 2,000,000 แพ็กเก็ต
จากนั้นนำมาวิเคราะห์เปรียบเทียบหาประสิทธิภาพด้านเวลาและการตรวจจับการโจมตี
\\\\
สาเหตุที่มีการทดลองสองรูปแบบนั้นก็เพื่อให้สามารถตรวจสอบได้ว่าผลลัพธ์ที่ได้จากการทดสอบมีความถูกต้องน่าเชื่อถือ
เป็นต้นว่าหากผลการทดสองจากทั้งสองรูปแบบสอดคล้องกัน ก็สามารถเชื่อได้ว่าผลการทดลองนั้นถูกต้อง นอกจากนั้นผู้ทำการทดลองยังได้กำหนดให้ลำดับวิธีของทั้ง 2 รูปแบบการทดลองมีความต่างกันในบางขั้นตอน เพื่อเพิ่มความเชื่อมั่นและความถูกต้องมากขั้นไปอีก
\\\\
\subsection{ขั้นตอนลำดับวิธีทดลอง}
\indent\indent
ในการทดลอง ขั้นตอนลำดับจะต่างกันในส่วนของรูปแบบการทดลองเท่านั้น โดยที่เงื่อนไขต่างๆ จะยังคงมีลำดับการทดลองที่เหมือนกัน กล่าวคือในเงื่อนไขการทดลองนั้นจะต่างกันก็เพียงแต่ไฟล์ที่ทำการรันเท่านั้น โดยที่ลำดับขั้นตอนจะยังคงเหมือนเดิม 
ในแต่ละเงิ่อนไขการทดลองจะทำการรันทดสอบ 3 รอบ และเลือกค่ากลางมาทำการวิเคราะห์ กล่าวคือ เลือกค่าที่ไม่น้อยที่สุดและไม่มากที่สุดจากค่าที่ได้จากการทดสอบ

\subsubsection*{รูปแบบการทดลองที่ 1 บน 1 \gls{controller}}

ในขั้นต้นผู้ทำการทดลองต้องทำการเปิด gnome terminal ขึ้นมาและเปิดหน้าต่างย่อยเป็นจำนวน 3 หน้าต่าง
ทุกหน้าต่างจะต้องทำการเชื่อมต่อไปยัง\gls{server}ด้วย ssh ให้เรียบร้อย
โดย หน้าต่างที่ 1 จะทำการรัน ryu-manager หน้าต่างที่ 2 จะทำการรัน mininet และหน้าต่างที่ 3 จะทำการรัน wireshark
โดยมีลำดับขั้นตอนการทดลองดังนี้


1. หน้าต่างที่ 1 ทำการรันคำสั่ง    
\$ ryu-manager learn-sdn-with-ryu/ryu-exercises/<python script> \\
โดย <python script> จะแทนด้วย ex7\_group\_tables.py หรือ ex8\_arp\_proxy.py ตามเงื่อนไขการทดลอง

2. หน่าต่างที่ 2 ทำการรันคำสั่ง    
\$ sudo python3 single\_topo.py \\
เพื่อทำการจำลอง topology แบบ 1 \gls{controller} ขึ้นมา

3. หน้าต่างที่ 3 ทำการรันคำสั่ง 
\$ sudo wireshark \\
โดยทำการจับแพ็กเก็ตที่ s4-eth1 (ที่เชื่อมกับ h2) และ s11-eth2 (ที่เชื่อมกับ h12) เป็นจำนวนที่เคยกล่าวไปข้างต้น

4. หลังจากที่ wireshark ทำการเริ่มจับแพ็กเก็ต หน้าต่างที่ 2 จะทำการรันคำสั่ง \$ mininet> h2 hping3 h12 -S --flood -V
โดยทันที เพื่อทำการให้ h2 โจมตีไปที่ h12

5. หลังจากทำการจับแพ็กเก็ตครบตามจำนวนที่กำหนด หน้าต่างที่ 2 ทำการหยุดการโจมตี จากนั้นทำการเก็บค่าสถิติ และทำการปิดโปรแกรม wireshark
หน้าต่างที่ 2 ทำการออกจาก mininet และรันคำสั่ง \$ sudo mn -c หากมีการทดสอบรอบต่อไปต้องเริ่มจากขั้นตอนที่ 1 ใหม่เท่านั้น

\subsubsection*{รูปแบบการทดลองที่ 1 บนหลาย\gls{controller}}

ในขั้นต้นผู้ทำการทดลองต้องทำการเปิด gnome terminal ขึ้นมาและเปิดหน้าต่างย่อยเป็นจำนวน 5 หน้าต่าง
ทุกหน้าต่างจะต้องทำการเชื่อมต่อไปยัง\gls{server}ด้วย ssh ให้เรียบร้อย
โดย หน้าต่างที่ 1 2 และ 3 จะทำการรัน ryu-manager หน้าต่างที่ 4 จะทำการรัน mininet และหน้าต่างที่ 5 จะทำการรัน wireshark
โดยมีลำดับขั้นตอนการทดลองดังนี้

1. หน้าต่างที่ 1 ทำการรันคำสั่ง \\
\$ ryu-manager learn-sdn-with-ryu/ryu-exercises/<python script> --ofp-tcp-listen-port 6633 \\
\indent หน้าต่างที่ 2 ทำการรันคำสั่ง \\
\$ ryu-manager learn-sdn-with-ryu/ryu-exercises/<python script> --ofp-tcp-listen-port 6634  \\
\indent หน้าต่างที่ 3 ทำการรันคำสั่ง \\
\$ ryu-manager learn-sdn-with-ryu/ryu-exercises/<python script> --ofp-tcp-listen-port 6635  \\
โดย <python script> จะแทนด้วย 
ex7\_group\_tables.py หรือ ex8\_arp\_proxy.py
ตามเงื่อนไขการทดลอง

2. หน่าต่างที่ 4 ทำการรันคำสั่ง    
\$ sudo python3 multi\_topo.py \\
เพื่อทำการจำลอง topology แบบหลาย\gls{controller} ขึ้นมา

3. หน้าต่างที่ 5 ทำการรันคำสั่ง 
\$ sudo wireshark 
โดยทำการจับแพ็กเก็ตที่ s4-eth1 (ที่เชื่อมกับ h2) และ s11-eth2 (ที่เชื่อมกับ h12) เป็นจำนวนที่เคยกล่าวไปข้างต้น

4. หลังจากที่ wireshark ทำการเริ่มจับแพ็กเก็ต หน้าต่างที่ 4 จะทำการรันคำสั่ง \$ mininet> h2 hping3 h12 -S --flood -V
โดยทันที เพื่อทำการให้ h2 โจมตีไปที่ h12

5. หลังจากทำการจับแพ็กเก็ตครบตามจำนวนที่กำหนด หน้าต่างที่ 4 ทำการหยุดการโจมตี จากนั้นทำการเก็บค่าสถิติ และทำการปิดโปรแกรม wireshark
หน้าต่างที่ 2 ทำการออกจาก mininet และรันคำสั่ง \$ sudo mn -c หากมีการทดสอบรอบต่อไปต้องเริ่มจากขั้นตอนที่ 1 ใหม่เท่านั้น

\subsubsection*{รูปแบบการทดลองที่ 2 บน 1 \gls{controller}}

ในขั้นต้นผู้ทำการทดลองต้องทำการเปิด gnome terminal ขึ้นมาและเปิดหน้าต่างย่อยเป็นจำนวน 3 หน้าต่าง
ทุกหน้าต่างจะต้องทำการเชื่อมต่อไปยัง\gls{server}ด้วย ssh ให้เรียบร้อย
โดย หน้าต่างที่ 1 จะทำการรัน ryu-manager หน้าต่างที่ 2 จะทำการรัน mininet และหน้าต่างที่ 3 จะทำการรัน wireshark
โดยมีลำดับขั้นตอนการทดลองดังนี้


1. หน้าต่างที่ 1 ทำการรันคำสั่ง    
\$ ryu-manager learn-sdn-with-ryu/ryu-exercises/<python script> \\
โดย <python script> จะแทนด้วย ex7\_group\_tables.py หรือ ex8\_arp\_proxy.py ตามเงื่อนไขการทดลอง

2. หน่าต่างที่ 2 ทำการรันคำสั่ง    
\$ sudo python3 single\_topo.py \\
เพื่อทำการจำลอง topology แบบ 1 \gls{controller} ขึ้นมา

3. หน้าต่างที่ 2 จะทำการรันคำสั่ง \$ mininet> h2 hping3 h12 -S --flood -V
เพื่อทำการให้ h2 โจมตีไปที่ h12

4. หน้าต่างที่ 3 ทำการรันคำสั่ง 
\$ sudo wireshark 
จากนั้นจะทำการเริ่มจับแพ็กเก็ตหลังจากเริ่มการโจมตีแล้ว 30 วินาที
โดยทำการจับแพ็กเก็ตที่ s4-eth1 (ที่เชื่อมกับ h2) และ s11-eth2 (ที่เชื่อมกับ h12) เป็นจำนวนที่เคยกล่าวไปข้างต้น


5. หลังจากทำการจับแพ็กเก็ตครบตามจำนวนที่กำหนด หน้าต่างที่ 2 ทำการหยุดการโจมตี จากนั้นทำการเก็บค่าสถิติ และทำการปิดโปรแกรม wireshark
หน้าต่างที่ 2 ทำการออกจาก mininet และรันคำสั่ง \$ sudo mn -c หากมีการทดสอบรอบต่อไปต้องเริ่มจากขั้นตอนที่ 1 ใหม่เท่านั้น

\subsubsection*{รูปแบบการทดลองที่ 2 บนหลาย\gls{controller}}

ในขั้นต้นผู้ทำการทดลองต้องทำการเปิด gnome terminal ขึ้นมาและเปิดหน้าต่างย่อยเป็นจำนวน 5 หน้าต่าง
ทุกหน้าต่างจะต้องทำการเชื่อมต่อไปยัง\gls{server}ด้วย ssh ให้เรียบร้อย
โดย หน้าต่างที่ 1 2 และ 3 จะทำการรัน ryu-manager หน้าต่างที่ 4 จะทำการรัน mininet และหน้าต่างที่ 5 จะทำการรัน wireshark
โดยมีลำดับขั้นตอนการทดลองดังนี้

1. หน้าต่างที่ 1 ทำการรันคำสั่ง \\
\$ ryu-manager learn-sdn-with-ryu/ryu-exercises/<python script> --ofp-tcp-listen-port 6633 \\
\indent หน้าต่างที่ 2 ทำการรันคำสั่ง \\
\$ ryu-manager learn-sdn-with-ryu/ryu-exercises/<python script> --ofp-tcp-listen-port 6634  \\
\indent หน้าต่างที่ 3 ทำการรันคำสั่ง \\
\$ ryu-manager learn-sdn-with-ryu/ryu-exercises/<python script> --ofp-tcp-listen-port 6635  \\
โดย <python script> จะแทนด้วย 
ex7\_group\_tables.py หรือ ex8\_arp\_proxy.py
ตามเงื่อนไขการทดลอง

2. หน่าต่างที่ 4 ทำการรันคำสั่ง    
\$ sudo python3 multi\_topo.py เพื่อทำการจำลอง topology แบบหลาย\gls{controller} ขึ้นมา

4. หน้าต่างที่ 4 จะทำการรันคำสั่ง \$ mininet> h2 hping3 h12 -S --flood -V
เพื่อทำการให้ h2 โจมตีไปที่ h12

3. หน้าต่างที่ 5 ทำการรันคำสั่ง 
\$ sudo wireshark 
จากนั้นจะทำการเริ่มจับแพ็กเก็ตหลังจากเริ่มการโจมตีแล้ว 30 วินาที
โดยทำการจับแพ็กเก็ตที่ s4-eth1 (ที่เชื่อมกับ h2) และ s11-eth2 (ที่เชื่อมกับ h12) เป็นจำนวนที่เคยกล่าวไปข้างต้น

5. หลังจากทำการจับแพ็กเก็ตครบตามจำนวนที่กำหนด หน้าต่างที่ 4 ทำการหยุดการโจมตี จากนั้นทำการเก็บค่าสถิติ และทำการปิดโปรแกรม wireshark
หน้าต่างที่ 2 ทำการออกจาก mininet และรันคำสั่ง \$ sudo mn -c หากมีการทดสอบรอบต่อไปต้องเริ่มจากขั้นตอนที่ 1 ใหม่เท่านั้น
\\\\
